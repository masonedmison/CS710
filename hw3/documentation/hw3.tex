\documentclass[11pt]{article}
\usepackage[margin = 1in]{geometry}
\usepackage[none]{hyphenat}
\usepackage{fancyhdr}
\usepackage{graphicx}
\usepackage{float}


\pagestyle{fancy}
\fancyhead{}
\fancyfoot{}
\fancyhead[L]{\slshape \MakeUppercase{Homework 2}}
\fancyhead[R]{\slshape Mason Edmison}
\fancyfoot[C]{\thepage}

\begin{document}

\begin{titlepage}
\begin{center}
\Large{\textbf{CS 710 - Artificial Intelligence}} \\
\vfill
\line(1,0){400} \\
\huge{\textbf{Homework 2}} \\
\Large{\textbf{Optimization and Classification}} \\
\line(1,0){400}\\
\vfill
Mason Edmison\\
University of Wisconsin-Milwaukee\\
10/22/2019
\end{center}
\end{titlepage}

\section{Question 1}
\subsection{}
Question 1a

Represent the following logic puzzle in First Order Logic using the predicates $Guilty(x)$ and $OutOfTown(x)$.  Then, provide a FOL resolution proof using answer extraction to determine who the guilty person is. Note that, with answer extraction,  unlike a normal refutation proof, you do NOT assert that Chloe is not guilty and find a contradiction! Instead, you assert $$\lnot Guilty(x) \lor Answer(x)$$, and create a proof that derives [$Answer(PERSON)]$ where $PERSON$ is the name of the guilty person.  Also, since non-guilty suspects tell the truth, their testimony can be represented by implications, for example, "Anne tells the police that Betty is innocent" can be represented as $$\lnot Guilty(A) \Rightarrow  \lnot Guilty(B)$$

The suspects in a robbery are Anne, Betty, and Chloe.
Exactly one of the suspects is guilty.
When questioned, a guilty suspect might lie or tell the truth, but an innocent one always tells the truth.
Anne tells the police that Betty is innocent.
Betty tells them that she was out of town the day the robbery occurred.
Chloe says that Betty was in town the day of the robbery.
If a suspect is out of town the day of the robbery, then she must be innocent.

\subsection{}


\section{Files of interest in this directory}
\begin{itemize}
\item \texttt{example.py} - Run for budget optimization - uses GA implementation listed below

\end{itemize}
\end{document}